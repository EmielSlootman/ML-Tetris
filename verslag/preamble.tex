% Allows you to insert special characters (UTF8)
\usepackage[utf8]{inputenc}

% Language used in the document.
\usepackage[english]{babel}


% -- Pictures
% Allows to insert pictures
\usepackage{graphicx}

% Allows to force positioning of pictures
\usepackage{float}

% Load eps files (MATLAB files)
\usepackage{epstopdf}


% -- Mathemtatics
% Additional mathematics
\usepackage{amsmath}

% Shortcuts for many physics related things.
% CAUTION: YOU MUST HAVE MCODE.STY FILE IN YOUR PROJECT
\usepackage{physics}

% Display measurement data easily
\usepackage{siunitx}
\sisetup{separate-uncertainty=true, multi-part-units=single, exponent-product=\cdot}

% -- Layout
% Change page size to use more normal margins
\usepackage{fullpage}

% Change how paragraphs are formatted
\usepackage{parskip}

% Allows you to make two of three column documents (e.g. papers)
\usepackage{multicol}

% Insert URLs
\usepackage[hidelinks]{hyperref}


% -- Organisation
% Allows you to use subfiles (very useful!)
\usepackage{subfiles}


% -- References
% Use the bibLaTeX package.
% CAUTION: YOU MUST ADDITIONALLY TELL LATEX WHICH LIBRARY YOU ARE USING WITH:
%          \addbibresource{main.bib}
%
%          Afterwards you can create the bibliography using
%          \printbibliography[heading=bibintoc,title={References}]
%          (The bibintoc adds the references to the table of contents)
\usepackage{biblatex}
